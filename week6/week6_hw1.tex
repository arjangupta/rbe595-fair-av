\documentclass{article}

\usepackage{fancyhdr}
\usepackage{extramarks}
\usepackage{amsmath}
\usepackage{amsthm}
\usepackage{amsfonts}
\usepackage{tikz}
\usepackage[plain]{algorithm}
\usepackage{algpseudocode}
\usepackage{graphicx}
\usepackage{gensymb}
\usepackage{hyperref}
\usepackage{enumitem}
\usepackage{listings}
\usepackage{siunitx}
\usepackage{matlab-prettifier}

\DeclareRobustCommand{\bbone}{\text{\usefont{U}{bbold}{m}{n}1}}

\DeclareMathOperator{\EX}{\mathbb{E}}% expected value

\graphicspath{{./images/}}

\usetikzlibrary{automata,positioning}

%
% Basic Document Settings
%

\topmargin=-0.45in
\evensidemargin=0in
\oddsidemargin=0in
\textwidth=6.5in
\textheight=9.0in
\headsep=0.25in

\linespread{1.1}

\pagestyle{fancy}
\lhead{\hmwkAuthorName}
\chead{\hmwkClassShort\ \hmwkTitle}
\rhead{\firstxmark}
\lfoot{\lastxmark}
\cfoot{\thepage}

\renewcommand\headrulewidth{0.4pt}
\renewcommand\footrulewidth{0.4pt}

\setlength\parindent{0pt}

%
% Create Problem Sections
%

\newcommand{\enterProblemHeader}[1]{
    \nobreak\extramarks{}{Problem {#1} continued on next page\ldots}\nobreak{}
    \nobreak\extramarks{{#1} (continued)}{{#1} continued on next page\ldots}\nobreak{}
}

\newcommand{\exitProblemHeader}[1]{
    \nobreak\extramarks{{#1} (continued)}{{#1} continued on next page\ldots}\nobreak{}
    % \stepcounter{#1}
    \nobreak\extramarks{{#1}}{}\nobreak{}
}

\setcounter{secnumdepth}{0}
\newcounter{partCounter}

\newcommand{\problemNumber}{0.0}

\newenvironment{homeworkProblem}[1][-1]{
    \renewcommand{\problemNumber}{{#1}}
    \section{\problemNumber}
    \setcounter{partCounter}{1}
    \enterProblemHeader{\problemNumber}
}{
    \exitProblemHeader{\problemNumber}
}

%
% Homework Details
%   - Title
%   - Class
%   - Author
%

\newcommand{\hmwkTitle}{Week 6 Homework \#1}
\newcommand{\hmwkClassShort}{RBE 595 (FAIR-AV)}
\newcommand{\hmwkClass}{RBE 595 --- FAIR-AV}
\newcommand{\hmwkAuthorName}{\textbf{Arjan Gupta}}

%
% Title Page
%

\title{
    \vspace{2in}
    \textmd{\textbf{\hmwkClass}}\\
    % \textmd{\textbf{\hmwkTitle}}\\
    \textmd{\textbf{Week 6 Homework \#1}}\\
    \vspace{3in}
}

\author{\hmwkAuthorName}
\date{}

\renewcommand{\part}[1]{\textbf{\large Part \Alph{partCounter}}\stepcounter{partCounter}\\}

%
% Various Helper Commands
%

% Useful for algorithms
\newcommand{\alg}[1]{\textsc{\bfseries \footnotesize #1}}

% For derivatives
\newcommand{\deriv}[2]{\frac{\mathrm{d}}{\mathrm{d}#2} \left(#1\right)}

% For compact derivatives
\newcommand{\derivcomp}[2]{\frac{\mathrm{d}#1}{\mathrm{d}#2}}

% For partial derivatives
\newcommand{\pderiv}[2]{\frac{\partial}{\partial #2} \left(#1\right)}

% For compact partial derivatives
\newcommand{\pderivcomp}[2]{\frac{\partial #1}{\partial #2}}

% Integral dx
\newcommand{\dx}{\mathrm{d}x}

% Alias for the Solution section header
\newcommand{\solution}{\textbf{\large Solution}}

% Probability commands: Expectation, Variance, Covariance, Bias
\newcommand{\E}{\mathrm{E}}
\newcommand{\Var}{\mathrm{Var}}
\newcommand{\Cov}{\mathrm{Cov}}
\newcommand{\Bias}{\mathrm{Bias}}

\begin{document}

\maketitle

\nobreak\extramarks{Problem 1}{}\nobreak{}

\pagebreak

\begin{homeworkProblem}[Problem 1]

    Let's define the time to accelerate a vehicle from 0 speed to 100 km/h as $\tau_{100}$, 
    which is constantly used by battery-electric car (BEV) manufacturers to brag about
    their technological advancement. 
    Provide the formulas to compute the minimal $\tau_{100}$ for

    \begin{enumerate}[label=(\alph*)]
        \item a four-wheel drive car (with an electric motor installed at each of the axles)
        \item a front-wheel drive (with an electric motor installed at the front axle)
        \item rear-wheel drive (with an electric motor installed at the rear axle)
    \end{enumerate}

    \section{Solution}

    The minimum time to accelerate a vehicle from 0 speed to 100 km/h can be found using the definite
    integral of the maximum acceleration that the vehicle can achieve. This would be given by
    the velocity-time relationship for the vehicle, which in this case would be

    \begin{equation}
        100 \, \text{km/h} = \int_{0}^{\tau_{100}} a(t) \, dt
    \end{equation}

    assuming that the vehicle starts from rest. The maximum acceleration that the vehicle can achieve
    is dependent on the configuration of the vehicle, as well as the road gradient.

    \subsection{Four-Wheel Drive}

    As per the lecture slides, the maximum acceleration that a vehicle can achieve in 4WD
    configuration is given by

    % v_dot max / g  = +/- mu cos alpha - sin alpha
    \begin{equation}
        a_{\text{max}} = g \left( \mu \cos(\alpha) - \sin(\alpha) \right)
    \end{equation}

    where $g$ is the acceleration due to gravity, $\mu$ is the coefficient of friction, and $\alpha$
    is the road gradient.

    Substituting this into the integral in equation (1), we get
    
    \begin{equation}
        100 = \int_{0}^{\tau_{100}} g \left( \mu \cos(\alpha) - \sin(\alpha) \right) \, dt
    \end{equation}

    Since everything in the integrand is constant, we can pull it out of the integral to get

    \begin{equation}
        100 = g \left( \mu \cos(\alpha) - \sin(\alpha) \right) \int_{0}^{\tau_{100}} \, dt
    \end{equation}

    Therefore, the minimum time to accelerate a 4WD vehicle from 0 speed to 100 km/h is

    \begin{equation}
        \tau_{100} = \frac{100}{g \left( \mu \cos(\alpha) - \sin(\alpha) \right)}
    \end{equation}

    \subsection{Front-Wheel Drive}

    The maximum acceleration that a vehicle can achieve in front-wheel drive configuration is given by

    \begin{equation}
        a_{\text{max}} = g \left( \frac{\mu}{1 + \mu \frac{h}{a_1 + a_2}} \frac{a_2}{a_1 + a_2} \right)
    \end{equation}

    


\end{homeworkProblem}

\end{document}