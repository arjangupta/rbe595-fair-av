\documentclass{article}

\usepackage{fancyhdr}
\usepackage{extramarks}
\usepackage{amsmath}
\usepackage{amsthm}
\usepackage{amsfonts}
\usepackage{tikz}
\usepackage[plain]{algorithm}
\usepackage{algpseudocode}
\usepackage{graphicx}
\usepackage{gensymb}
\usepackage{hyperref}
\usepackage{enumitem}
\usepackage{listings}
\usepackage{siunitx}
\usepackage{matlab-prettifier}

\DeclareRobustCommand{\bbone}{\text{\usefont{U}{bbold}{m}{n}1}}

\DeclareMathOperator{\EX}{\mathbb{E}}% expected value

\graphicspath{{./images/}}

\usetikzlibrary{automata,positioning}

%
% Basic Document Settings
%

\topmargin=-0.45in
\evensidemargin=0in
\oddsidemargin=0in
\textwidth=6.5in
\textheight=9.0in
\headsep=0.25in

\linespread{1.1}

\pagestyle{fancy}
\lhead{\hmwkAuthorName}
\chead{\hmwkClassShort\ \hmwkTitle}
\rhead{\firstxmark}
\lfoot{\lastxmark}
\cfoot{\thepage}

\renewcommand\headrulewidth{0.4pt}
\renewcommand\footrulewidth{0.4pt}

\setlength\parindent{0pt}

%
% Create Problem Sections
%

\newcommand{\enterProblemHeader}[1]{
    \nobreak\extramarks{}{Problem {#1} continued on next page\ldots}\nobreak{}
    \nobreak\extramarks{{#1} (continued)}{{#1} continued on next page\ldots}\nobreak{}
}

\newcommand{\exitProblemHeader}[1]{
    \nobreak\extramarks{{#1} (continued)}{{#1} continued on next page\ldots}\nobreak{}
    % \stepcounter{#1}
    \nobreak\extramarks{{#1}}{}\nobreak{}
}

\setcounter{secnumdepth}{0}
\newcounter{partCounter}

\newcommand{\problemNumber}{0.0}

\newenvironment{homeworkProblem}[1][-1]{
    \renewcommand{\problemNumber}{{#1}}
    \section{\problemNumber}
    \setcounter{partCounter}{1}
    \enterProblemHeader{\problemNumber}
}{
    \exitProblemHeader{\problemNumber}
}

%
% Homework Details
%   - Title
%   - Class
%   - Author
%

\newcommand{\hmwkTitle}{Week 6 Homework \#2}
\newcommand{\hmwkClassShort}{RBE 595 (FAIR-AV)}
\newcommand{\hmwkClass}{RBE 595 --- FAIR-AV}
\newcommand{\hmwkAuthorName}{\textbf{Arjan Gupta}}

%
% Title Page
%

\title{
    \vspace{2in}
    \textmd{\textbf{\hmwkClass}}\\
    % \textmd{\textbf{\hmwkTitle}}\\
    \textmd{\textbf{Week 6 Homework \#2}}\\
    \vspace{3in}
}

\author{\hmwkAuthorName}
\date{}

\renewcommand{\part}[1]{\textbf{\large Part \Alph{partCounter}}\stepcounter{partCounter}\\}

%
% Various Helper Commands
%

% Useful for algorithms
\newcommand{\alg}[1]{\textsc{\bfseries \footnotesize #1}}

% For derivatives
\newcommand{\deriv}[2]{\frac{\mathrm{d}}{\mathrm{d}#2} \left(#1\right)}

% For compact derivatives
\newcommand{\derivcomp}[2]{\frac{\mathrm{d}#1}{\mathrm{d}#2}}

% For partial derivatives
\newcommand{\pderiv}[2]{\frac{\partial}{\partial #2} \left(#1\right)}

% For compact partial derivatives
\newcommand{\pderivcomp}[2]{\frac{\partial #1}{\partial #2}}

% Integral dx
\newcommand{\dx}{\mathrm{d}x}

% Alias for the Solution section header
\newcommand{\solution}{\textbf{\large Solution}}

% Probability commands: Expectation, Variance, Covariance, Bias
\newcommand{\E}{\mathrm{E}}
\newcommand{\Var}{\mathrm{Var}}
\newcommand{\Cov}{\mathrm{Cov}}
\newcommand{\Bias}{\mathrm{Bias}}

\begin{document}

\maketitle

\nobreak\extramarks{Problem 1}{}\nobreak{}

\pagebreak

\begin{homeworkProblem}[Problem 1]

    Assume your AV has been officially rolled out in a hilly city like San Francisco. 
    But suddenly you find that, in some of the steep streets, your AVs cannot parked well, 
    namely, the AV would roll away when parked.  
    What is the quickest way to fix the problem with minimum cost? 

    \section{Solution}

    One very simple and quick way to fix this problem would be
    to issue a software update that will use the AV's on-board inclinometer 
    readings to avoid parking on steep streets, or turn the wheels in a way that
    the AV will not roll away when parked.
    The software update could
    be as simple as a threshold check on the inclinometer readings, and
    if the readings exceed a certain threshold, the AV will not park
    on that street, or will turn the wheels in a way that the AV will not roll away.
    This solution is quick and can be implemented with minimal cost, as it
    only requires a software update to the AVs.
    \vspace{0.5cm}

    However, this solution is not sustainable in the long run, as avoiding
    steep streets would limit parking options for the AVs, and turning the
    wheels in a way that the AV will not roll away may not be a foolproof
    solution. A more sustainable solution would be to design the AVs with
    a parking brake system that can be engaged when parked on steep streets.
    \vspace{0.5cm}

    The parking brake system would be designed to engage when the AV is parked on a steep street,
    preventing the AV from rolling away. The parking brake system could be activated
    automatically by the AV's software when the inclinometer readings
    exceed a certain threshold. This would make it so that the angle $\alpha$ of the street
    would be taken into account when parking the AV, as shown by the following equation:

    \begin{equation}
        - \frac{a_1}{h} \cos(\alpha) \leq \frac{\dot{v}}{g} + \sin(\alpha) \leq \frac{a_2}{h} \cos(\alpha)
    \end{equation}

    where $a_1$ and $a_2$ are the minimum and maximum acceleration of the AV, $h$ is the height of the center of mass of the AV, $\dot{v}$ is the velocity of the AV, and $g$ is the acceleration due to gravity.
    \vspace{0.5cm}


\end{homeworkProblem}

\end{document}