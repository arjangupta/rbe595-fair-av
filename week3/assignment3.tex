\documentclass{article}

\usepackage{fancyhdr}
\usepackage{extramarks}
\usepackage{amsmath}
\usepackage{amsthm}
\usepackage{amsfonts}
\usepackage{tikz}
\usepackage[plain]{algorithm}
\usepackage{algpseudocode}
\usepackage{graphicx}
\usepackage{gensymb}
\usepackage{hyperref}
\usepackage{enumitem}
\usepackage{listings}
\usepackage{siunitx}
\usepackage{matlab-prettifier}

\DeclareRobustCommand{\bbone}{\text{\usefont{U}{bbold}{m}{n}1}}

\DeclareMathOperator{\EX}{\mathbb{E}}% expected value

\graphicspath{{./images/}}

\usetikzlibrary{automata,positioning}

%
% Basic Document Settings
%

\topmargin=-0.45in
\evensidemargin=0in
\oddsidemargin=0in
\textwidth=6.5in
\textheight=9.0in
\headsep=0.25in

\linespread{1.1}

\pagestyle{fancy}
\lhead{\hmwkAuthorName}
\chead{\hmwkClassShort\ \hmwkTitle}
\rhead{\firstxmark}
\lfoot{\lastxmark}
\cfoot{\thepage}

\renewcommand\headrulewidth{0.4pt}
\renewcommand\footrulewidth{0.4pt}

\setlength\parindent{0pt}

%
% Create Problem Sections
%

\newcommand{\enterProblemHeader}[1]{
    \nobreak\extramarks{}{Problem {#1} continued on next page\ldots}\nobreak{}
    \nobreak\extramarks{{#1} (continued)}{{#1} continued on next page\ldots}\nobreak{}
}

\newcommand{\exitProblemHeader}[1]{
    \nobreak\extramarks{{#1} (continued)}{{#1} continued on next page\ldots}\nobreak{}
    % \stepcounter{#1}
    \nobreak\extramarks{{#1}}{}\nobreak{}
}

\setcounter{secnumdepth}{0}
\newcounter{partCounter}

\newcommand{\problemNumber}{0.0}

\newenvironment{homeworkProblem}[1][-1]{
    \renewcommand{\problemNumber}{{#1}}
    \section{\problemNumber}
    \setcounter{partCounter}{1}
    \enterProblemHeader{\problemNumber}
}{
    \exitProblemHeader{\problemNumber}
}

%
% Homework Details
%   - Title
%   - Class
%   - Author
%

\newcommand{\hmwkTitle}{Assignment \#3}
\newcommand{\hmwkClassShort}{RBE 595 (FAIR-AV)}
\newcommand{\hmwkClass}{RBE 595 --- FAIR-AV}
\newcommand{\hmwkAuthorName}{\textbf{Arjan Gupta}}

%
% Title Page
%

\title{
    \vspace{2in}
    \textmd{\textbf{\hmwkClass}}\\
    % \textmd{\textbf{\hmwkTitle}}\\
    \textmd{\textbf{Assignment \#3}}\\
    \vspace{3in}
}

\author{\hmwkAuthorName}
\date{}

\renewcommand{\part}[1]{\textbf{\large Part \Alph{partCounter}}\stepcounter{partCounter}\\}

%
% Various Helper Commands
%

% Useful for algorithms
\newcommand{\alg}[1]{\textsc{\bfseries \footnotesize #1}}

% For derivatives
\newcommand{\deriv}[2]{\frac{\mathrm{d}}{\mathrm{d}#2} \left(#1\right)}

% For compact derivatives
\newcommand{\derivcomp}[2]{\frac{\mathrm{d}#1}{\mathrm{d}#2}}

% For partial derivatives
\newcommand{\pderiv}[2]{\frac{\partial}{\partial #2} \left(#1\right)}

% For compact partial derivatives
\newcommand{\pderivcomp}[2]{\frac{\partial #1}{\partial #2}}

% Integral dx
\newcommand{\dx}{\mathrm{d}x}

% Alias for the Solution section header
\newcommand{\solution}{\textbf{\large Solution}}

% Probability commands: Expectation, Variance, Covariance, Bias
\newcommand{\E}{\mathrm{E}}
\newcommand{\Var}{\mathrm{Var}}
\newcommand{\Cov}{\mathrm{Cov}}
\newcommand{\Bias}{\mathrm{Bias}}

\begin{document}

\maketitle

\nobreak\extramarks{Problem 1}{}\nobreak{}

\pagebreak

\begin{homeworkProblem}[Problem 1]

    Read The One Hundred Year Study on Artificial Intelligence (AI100) 2021 Study Panel Report.
    Based on what you have read and your understanding of AV, provide a one-page summary that covers answers to the following two questions:

    how AV is challenging AI? (3pt)

    how AV is providing opportunities for AI? (2pt)

    \section{Solution}

    \subsection{How AV is challenging AI}

    Autonomous Vehicles (AVs) present a unique set of challenges to the field of Artificial Intelligence
    (AI). The AI100 report first mentions autonomous vehicles on page 7, where
    it is noted that ``predicted rapid progress in fully autonomous driving failed to 
    materialize'', which can help us understand the main challenge that AVs pose to AI research.
    In fact in the Mobility section on page 16,
    it is mentioned that the reasons for this failure in materialization are complicated, 
    but the need for exceptional levels of safety in complex physical environments makes
    the problem more challenging.
    In the same Mobility section, it is also mentioned that the
    design of autonomous vehicles requires the integration of several technologies, some of which
    are AI planning and decision-making.
    Hence, one of the biggest challenges that AVs pose to AI research is safety --- or in other
    words, the need
    to develop better and more advanced AI systems that can operate safely and 
    reliably in a wide range of environments and conditions. This attention to safety
    is reiterated on page 18 of the report, where it is mentioned that one of the grand challenges
    of AI is to provide an ``accident-avoiding car''.
    \vspace{1em}

    In my opinion, to make AVs become more widespread, the safety challenges that they pose
    to AI research will only increase. AVs will need to be able to operate in a wide range of
    environments, including urban areas, highways, and rural roads, and they will need to be
    able to handle a wide range of weather and traffic conditions. This will require AI systems
    that are capable of learning and adapting to new situations, and that are able to make
    decisions quickly, safely, and accurately. In addition, AVs will need to be able to communicate
    safety data with other vehicles and with infrastructure systems, which will require AI systems that
    are capable of understanding and interpreting complex data and information. Overall, the
    challenges that AVs pose to AI research are significant, but they also present an opportunity
    for AI to develop new and innovative solutions to complex problems.

    \subsection{How AV is providing opportunities for AI}

    A major opportunity that AVs provide to AI research is the need to develop AI systems that
    can greatly assist human drivers, since replacing humans cannot be done overnight. Eventually AVs with
    advanced AI capabilities will eliminate
    the need for human drivers completely, but there will be an interim period where many
    humans can benefit from additional AI in their cars to assist them. On page 49 of the AI100 Report, 
    under the section
    about assisting with decision-making, it is mentioned that AI systems are good for keeping the vehicle
    in lane and watching for sudden changes in traffic flow, while the human is better equipped for making
    major route decisions and watching for hazards. This leads into AI as an assistant to human drivers,
    which is an opportunity detailed on page 51. This section about ``AI as an assistant'', which is
    mentioned as an area of opportunity for augmentation of AI, mentions that AI
    can help with tasks like lane-keeping assistance and reaction-support systems. In fact there are
    many other subsystems helping create this AI-assistant capabilities, such as advanced computer vision
    systems like YOLO (mentioned on page 14). YOLO is a real-time object detection system that can
    be used to detect and track objects in the environment, which is a key capability for AVs.
    \vspace{1em}

    In my opinion, as AI advances, it can assist with more and more tasks that human drivers currently
    perform, and this will lead to a safer and more efficient driving experience for everyone. Examples
    of this can be seen in the lectures from this week. For example, perception AI can be used to
    provide situational awareness around the vehicle, night vision, traffic jam pilot, 
    and anomaly detection features. These are all examples of how AI can assist human drivers
    in making better decisions and driving more safely. Overall, the opportunities that AVs provide
    to AI research are significant, and they have the potential to lead to new and innovative
    solutions to complex problems in the field of AI.

\end{homeworkProblem}

\end{document}