\documentclass{article}

\usepackage{fancyhdr}
\usepackage{extramarks}
\usepackage{amsmath}
\usepackage{amsthm}
\usepackage{amsfonts}
\usepackage{tikz}
\usepackage[plain]{algorithm}
\usepackage{algpseudocode}
\usepackage{graphicx}
\usepackage{gensymb}
\usepackage{hyperref}
\usepackage{enumitem}
\usepackage{listings}
\usepackage{siunitx}
\usepackage{matlab-prettifier}

\DeclareRobustCommand{\bbone}{\text{\usefont{U}{bbold}{m}{n}1}}

\DeclareMathOperator{\EX}{\mathbb{E}}% expected value

\graphicspath{{./images/}}

\usetikzlibrary{automata,positioning}

%
% Basic Document Settings
%

\topmargin=-0.45in
\evensidemargin=0in
\oddsidemargin=0in
\textwidth=6.5in
\textheight=9.0in
\headsep=0.25in

\linespread{1.1}

\pagestyle{fancy}
\lhead{\hmwkAuthorName}
\chead{\hmwkClassShort\ \hmwkTitle}
\rhead{\firstxmark}
\lfoot{\lastxmark}
\cfoot{\thepage}

\renewcommand\headrulewidth{0.4pt}
\renewcommand\footrulewidth{0.4pt}

\setlength\parindent{0pt}

%
% Create Problem Sections
%

\newcommand{\enterProblemHeader}[1]{
    \nobreak\extramarks{}{Problem {#1} continued on next page\ldots}\nobreak{}
    \nobreak\extramarks{{#1} (continued)}{{#1} continued on next page\ldots}\nobreak{}
}

\newcommand{\exitProblemHeader}[1]{
    \nobreak\extramarks{{#1} (continued)}{{#1} continued on next page\ldots}\nobreak{}
    % \stepcounter{#1}
    \nobreak\extramarks{{#1}}{}\nobreak{}
}

\setcounter{secnumdepth}{0}
\newcounter{partCounter}

\newcommand{\problemNumber}{0.0}

\newenvironment{homeworkProblem}[1][-1]{
    \renewcommand{\problemNumber}{{#1}}
    \section{\problemNumber}
    \setcounter{partCounter}{1}
    \enterProblemHeader{\problemNumber}
}{
    \exitProblemHeader{\problemNumber}
}

%
% Homework Details
%   - Title
%   - Class
%   - Author
%

\newcommand{\hmwkTitle}{Assignment \#1}
\newcommand{\hmwkClassShort}{RBE 595 (FAIR-AV)}
\newcommand{\hmwkClass}{RBE 595 --- FAIR-AV}
\newcommand{\hmwkAuthorName}{\textbf{Arjan Gupta}}

%
% Title Page
%

\title{
    \vspace{2in}
    \textmd{\textbf{\hmwkClass}}\\
    % \textmd{\textbf{\hmwkTitle}}\\
    \textmd{\textbf{Assignment \#1}}\\
    \vspace{3in}
}

\author{\hmwkAuthorName}
\date{}

\renewcommand{\part}[1]{\textbf{\large Part \Alph{partCounter}}\stepcounter{partCounter}\\}

%
% Various Helper Commands
%

% Useful for algorithms
\newcommand{\alg}[1]{\textsc{\bfseries \footnotesize #1}}

% For derivatives
\newcommand{\deriv}[2]{\frac{\mathrm{d}}{\mathrm{d}#2} \left(#1\right)}

% For compact derivatives
\newcommand{\derivcomp}[2]{\frac{\mathrm{d}#1}{\mathrm{d}#2}}

% For partial derivatives
\newcommand{\pderiv}[2]{\frac{\partial}{\partial #2} \left(#1\right)}

% For compact partial derivatives
\newcommand{\pderivcomp}[2]{\frac{\partial #1}{\partial #2}}

% Integral dx
\newcommand{\dx}{\mathrm{d}x}

% Alias for the Solution section header
\newcommand{\solution}{\textbf{\large Solution}}

% Probability commands: Expectation, Variance, Covariance, Bias
\newcommand{\E}{\mathrm{E}}
\newcommand{\Var}{\mathrm{Var}}
\newcommand{\Cov}{\mathrm{Cov}}
\newcommand{\Bias}{\mathrm{Bias}}

\begin{document}

\maketitle

\nobreak\extramarks{Problem 1}{}\nobreak{}

\pagebreak

\begin{homeworkProblem}[Problem 1]

    Use your favorite auto OEM as an example to compute the PCDM using its average ROI or last year's ROI. (2pt)

    \section{Solution}

    I have chosen Honda as an example. The ROI for Honda in 2023 is the average of the data points from 2023,
    per the data available on \href{https://www.macrotrends.net/stocks/charts/HMC/honda/roi}{Macrotrends}.
    These data points are 4.85\%, 5.85\%, 6.15\%, and 6.53\% respectively for the quarters of 2023. The average ROI
    for Honda in 2023 is given by,

    \begin{align*}
        \text{Average ROI} &= \frac{4.85 + 5.85 + 6.15 + 6.53}{4} \\
        &= 5.845\%
    \end{align*}

    The PCDM is the profit per customer driven mile. In 2023, Honda recorded an operating profit
    of approximately 781 billion Japanese yen. 
    With around 4.5 million cars sold, this translates to an average profit of about 
    173,555 yen (roughly \$1,200 USD) per car. The source for this data 
    is \href{https://www.statista.com/statistics/1126585/honda-operating-profit/}{Statista}.

    On average, Honda cars have a mileage lifespan of about 200,000 miles. The source for this data is
    \href{https://www.brickellhonda.com/the-longevity-of-honda-cars-mileage-maintenance-and-more}{Brickell Honda}.

    Therefore, we can calculate the PCDM for Honda in 2023 as follows,

    \begin{align*}
        \text{PCDM} &= \frac{\text{Average profit per car}}{\text{Average mileage per car}} \\
        &= \frac{\$1,200}{200,000} \\
        &= \$0.006 \text{ per mile}
    \end{align*}

    Let us compound the PCDM with the average ROI for Honda in 2023,

    \begin{align*}
        \text{PCDM} \times \text{(1 + ROI)}^{10} &= \$0.006 \times (1 + 0.05845)^{10} \\
        &= \$0.006 \times 1.765 \\
        &= \$0.01 \text{ per mile}
    \end{align*}

    Therefore, the PCDM for Honda in 2023, compounded with the average ROI, is \$0.01 per mile.

\end{homeworkProblem}

\nobreak\extramarks{Problem 2}{}\nobreak{}

\pagebreak

\begin{homeworkProblem}[Problem 2]

    If an Uber or Lyft driver uses his or her own self-driving car for the car-sharing service 
    (if it is allowed by the car-sharing company), what are the implications
    for the 4 agents discussed in lecture \#2? (2pt)

    \section{Solution}

    The four agents discussed in lecture \#2 are:

    \begin{enumerate}
        \item The car owner
        \item The customer using the car-sharing service
        \item The car-sharing company (Uber or Lyft)
        \item The car manufacturer
    \end{enumerate}

    If an Uber or Lyft driver uses his or her own self-driving car for the car-sharing service, the implications
    for the four agents are as follows:

    \begin{enumerate}
        \item \textbf{The car owner:} The car owner is the Uber or Lyft driver in this case. The car owner
        will benefit from the car-sharing service by earning money from the rides given by the self-driving car.
        The car owner will also save on the cost of hiring a driver, as the self-driving car can operate without
        human intervention. The car owner will also save on the cost of fuel, as the self-driving car can be
        programmed to drive efficiently.
        \item \textbf{The customer using the car-sharing service:} The customer will benefit from the car-sharing
        service by being able to book a ride from the self-driving car. The customer will also benefit from the
        convenience of being able to book a ride at any time, as the self-driving car can operate 24/7.
        \item \textbf{The car-sharing company (Uber or Lyft):} The car-sharing company will benefit from the
        self-driving car by being able to offer rides to customers without having to pay a driver. The car-sharing
        company will also benefit from the increased efficiency of the self-driving car, as it can operate
        continuously without breaks.
        \item \textbf{The car manufacturer:} The car manufacturer will benefit from the self-driving car by
        being able to sell more cars to Uber or Lyft drivers who want to use their own self-driving cars for
        the car-sharing service. The car manufacturer will also benefit from the increased demand for self-driving
        cars, as more people will want to use self-driving cars for the car-sharing service. However, some
        car manufacturers may not have self-driving cars in their product lineup, so they may lose out on
        sales to car manufacturers that do have self-driving cars, or those that have better self-driving
        technology.
    \end{enumerate}

    In conclusion, the implications of an Uber or Lyft driver using his or her own self-driving car for the
    car-sharing service are positive for all four agents discussed in lecture \#2, with the last agent
    having a potential downside if they do not have self-driving cars in their product lineup.

\end{homeworkProblem}

\nobreak\extramarks{Problem 3}{}\nobreak{}

\pagebreak

\begin{homeworkProblem}[Problem 3]
    If Uber or Lyft owns self-driving cars, what are the implications? (2pt)

    \section{Solution}

    For the same four agents discussed in lecture \#2, the implications of Uber or Lyft owning self-driving cars
    are as follows:

    \begin{enumerate}
        \item \textbf{The car owner:} The car owner in this case is Uber or Lyft. Uber or Lyft will benefit
        from owning self-driving cars by being able to offer rides to customers without having to pay a driver.
        Uber or Lyft will also benefit from the increased efficiency of the self-driving cars, as they can
        operate continuously without breaks. However, Uber or Lyft will have to bear the cost of purchasing,
        maintaining, and operating the self-driving cars, and doing this at scale can be complex and expensive.
        \item \textbf{The customer using the car-sharing service:} The customer will benefit from the car-sharing
        service by being able to book a ride from the self-driving car. The customer will also benefit from the
        convenience of being able to book a ride at any time, as the self-driving car can operate 24/7. The customer
        may also benefit from lower prices, as Uber or Lyft may pass on the cost savings from not having to pay
        a driver to the customer.
        \item \textbf{The car-sharing company (Uber or Lyft):} In this case, the owner and the car-sharing company
        are the same, so this point is the same as the first point.
        \item \textbf{The car manufacturer:} The car manufacturer will benefit from Uber or Lyft owning self-driving
        cars by being able to sell more cars to Uber or Lyft. The car manufacturer will also benefit from the
        increased demand for self-driving cars, as more people will want to use self-driving cars for the car-sharing
        service. However, some car manufacturers may not have self-driving cars in their product lineup, so they
        may lose out on sales to car manufacturers that do have self-driving cars, or those that have better
        self-driving technology.
    \end{enumerate}

\end{homeworkProblem}

\end{document}