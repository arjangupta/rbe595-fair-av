\documentclass{article}

\usepackage{fancyhdr}
\usepackage{extramarks}
\usepackage{amsmath}
\usepackage{amsthm}
\usepackage{amsfonts}
\usepackage{tikz}
\usepackage[plain]{algorithm}
\usepackage{algpseudocode}
\usepackage{graphicx}
\usepackage{gensymb}
\usepackage{hyperref}
\usepackage{enumitem}
\usepackage{listings}
\usepackage{siunitx}
\usepackage{matlab-prettifier}

\DeclareRobustCommand{\bbone}{\text{\usefont{U}{bbold}{m}{n}1}}

\DeclareMathOperator{\EX}{\mathbb{E}}% expected value

\graphicspath{{./images/}}

\usetikzlibrary{automata,positioning}

%
% Basic Document Settings
%

\topmargin=-0.45in
\evensidemargin=0in
\oddsidemargin=0in
\textwidth=6.5in
\textheight=9.0in
\headsep=0.25in

\linespread{1.1}

\pagestyle{fancy}
\lhead{\hmwkAuthorName}
\chead{\hmwkClassShort\ \hmwkTitle}
\rhead{\firstxmark}
\lfoot{\lastxmark}
\cfoot{\thepage}

\renewcommand\headrulewidth{0.4pt}
\renewcommand\footrulewidth{0.4pt}

\setlength\parindent{0pt}

%
% Create Problem Sections
%

\newcommand{\enterProblemHeader}[1]{
    \nobreak\extramarks{}{Problem {#1} continued on next page\ldots}\nobreak{}
    \nobreak\extramarks{{#1} (continued)}{{#1} continued on next page\ldots}\nobreak{}
}

\newcommand{\exitProblemHeader}[1]{
    \nobreak\extramarks{{#1} (continued)}{{#1} continued on next page\ldots}\nobreak{}
    % \stepcounter{#1}
    \nobreak\extramarks{{#1}}{}\nobreak{}
}

\setcounter{secnumdepth}{0}
\newcounter{partCounter}

\newcommand{\problemNumber}{0.0}

\newenvironment{homeworkProblem}[1][-1]{
    \renewcommand{\problemNumber}{{#1}}
    \section{\problemNumber}
    \setcounter{partCounter}{1}
    \enterProblemHeader{\problemNumber}
}{
    \exitProblemHeader{\problemNumber}
}

%
% Homework Details
%   - Title
%   - Class
%   - Author
%

\newcommand{\hmwkTitle}{Assignment \#2}
\newcommand{\hmwkClassShort}{RBE 595 (FAIR-AV)}
\newcommand{\hmwkClass}{RBE 595 --- FAIR-AV}
\newcommand{\hmwkAuthorName}{\textbf{Arjan Gupta}}

%
% Title Page
%

\title{
    \vspace{2in}
    \textmd{\textbf{\hmwkClass}}\\
    % \textmd{\textbf{\hmwkTitle}}\\
    \textmd{\textbf{Assignment \#2}}\\
    \vspace{3in}
}

\author{\hmwkAuthorName}
\date{}

\renewcommand{\part}[1]{\textbf{\large Part \Alph{partCounter}}\stepcounter{partCounter}\\}

%
% Various Helper Commands
%

% Useful for algorithms
\newcommand{\alg}[1]{\textsc{\bfseries \footnotesize #1}}

% For derivatives
\newcommand{\deriv}[2]{\frac{\mathrm{d}}{\mathrm{d}#2} \left(#1\right)}

% For compact derivatives
\newcommand{\derivcomp}[2]{\frac{\mathrm{d}#1}{\mathrm{d}#2}}

% For partial derivatives
\newcommand{\pderiv}[2]{\frac{\partial}{\partial #2} \left(#1\right)}

% For compact partial derivatives
\newcommand{\pderivcomp}[2]{\frac{\partial #1}{\partial #2}}

% Integral dx
\newcommand{\dx}{\mathrm{d}x}

% Alias for the Solution section header
\newcommand{\solution}{\textbf{\large Solution}}

% Probability commands: Expectation, Variance, Covariance, Bias
\newcommand{\E}{\mathrm{E}}
\newcommand{\Var}{\mathrm{Var}}
\newcommand{\Cov}{\mathrm{Cov}}
\newcommand{\Bias}{\mathrm{Bias}}

\begin{document}

\maketitle

\nobreak\extramarks{Problem 1}{}\nobreak{}

\pagebreak

\begin{homeworkProblem}[Problem 1]

    If the digital driver or virtual driver is used to control an autonomous vehicle, what changes in the vehicle have to be made?

    Hint \#1: Think of the framework of shared control and replace the human driver with a digital driver.

    Hint \#2: The changes in the vehicle could be from hardware, software, and system architecture. 

    \section{Solution}

    The digital driver or virtual driver is a software system that is responsible 
    for controlling the vehicle. The digital driver is responsible for making
    decisions about the vehicle's control inputs, such as steering, throttle, and
    braking. The digital driver is also responsible for perceiving the environment
    around the vehicle and making decisions based on this perception.

    The digital driver is a replacement for the human driver in an autonomous vehicle.
    The digital driver is responsible for all aspects of driving the vehicle, including
    perception, planning, and control. The digital driver must be able to perceive the
    environment around the vehicle, plan a safe and efficient path through this environment,
    and control the vehicle to follow this path.

    In order to use a digital driver to control an autonomous vehicle, several changes
    must be made to the vehicle. These changes can be categorized into three main areas:
    hardware, software, and system architecture.

    \begin{enumerate}
        \item \textbf{Hardware Changes:} The hardware of the vehicle must be modified to
        support the digital driver. This may include the addition of sensors, such as cameras,
        lidar, radar, and ultrasonic sensors, to enable the digital driver to perceive the
        environment around the vehicle. The vehicle may also need additional computing power
        to run the digital driver software and process the sensor data in real-time. In addition,
        the vehicle may need actuators, such as electric motors, to control the vehicle's steering,
        throttle, and braking systems.

        \item \textbf{Software Changes:} The software of the vehicle must be modified to
        support the digital driver. This may include the development of perception, planning,
        and control algorithms to enable the digital driver to make decisions about the vehicle's
        control inputs. The software must also be able to interface with the vehicle's sensors
        and actuators to enable the digital driver to perceive the environment and control the
        vehicle in real-time.

        \item \textbf{System Architecture Changes:} The system architecture of the vehicle
        must be modified to support the digital driver. This may include the development of a
        distributed computing system to enable the digital driver to run on multiple processors
        and communicate with the vehicle's sensors and actuators. The system architecture must
        also be fault-tolerant to ensure the safe operation of the vehicle in the event of a
        software or hardware failure.
    \end{enumerate}

    In conclusion, the digital driver is a replacement for the human driver in an autonomous
    vehicle. In order to use a digital driver to control an autonomous vehicle, several changes
    must be made to the vehicle, including hardware, software, and system architecture changes.

\end{homeworkProblem}

\end{document}